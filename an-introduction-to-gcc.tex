\documentclass[lang=cn,10pt,newtx,scheme=chinese]{elegantbook}

\title{An Introduction to GCC: GCC简明介绍}
\subtitle{for the GNU Compilers \textbf{gcc} and \textbf{g++}}

\author{Brian Gough}
\date{March, 2004}
\bioinfo{Forword}{Richard M. Stallman}

\setcounter{tocdepth}{3}

\cover{cover.jpg}

% 本文档命令
\usepackage{array}
\newcommand{\ccr}[1]{\makecell{{\color{#1}\rule{1cm}{1cm}}}}

% 修改标题页的橙色带
\definecolor{customcolor}{RGB}{32,178,170}
\colorlet{coverlinecolor}{customcolor}
\usepackage{cprotect}

\addbibresource[location=local]{reference.bib} % 参考文献,不要删除


%==========================目录层次设置============================%
\setcounter{part}{-1}
\setcounter{chapter}{0}%设置目录计数初值
\setcounter{tocdepth}{3} %设置目录的显示级别。book类从-1开始,为第一级。
\setcounter{secnumdepth}{3}
%(1)设置自动编号的深度(即编号到哪一级别)。{num}在article中取0到5的整数相应于目录中显示到\chapter,\section,\subsection,\subsubsection,\paragraph 或\subparagraph 层次。在book和report中,num取-1到5之间的整数相应于在目录中显示到\part,\chapter,\section,\subseciton,\subsubsection, \paragraph 或\subparagraph 层次。          
%(2)对长标题用\section[abc]{abcdefg}形式的命令。                                            
%(3)另外,还可以利用\addtocounter{secnumdepth}{num}来使得当前章节编号深度增加或减小,num可取正值或负值。
%(4)对于高级内容要求的章节以星号标识,然后在正文中用\begin{advanced}\section{...}\end{advanced}即可
%------------------------------------------------------------------------------------------------------------------------%

\begin{document}

\maketitle
\frontmatter

\tableofcontents
\mainmatter

\chapter*{前言}
\addcontentsline{toc}{chapter}{前言}

\chapter{介绍}
\section{GCC的简要历史}
\section{GCC的主要特点}
\section{在C和C++中编程}
\section{本手册使用的惯例}

\chapter{编译一个 C 程序}
\section{编译一个简单的 C 程序}
\section{在一个简单的程序中寻找错误}
\section{编译多个源文件}
\section{独立编译文件}
\subsection{从源文件创建目标文件}
\subsection{从目标文件创建可执行文件}
\subsection{目标文件的链接顺序}
\section{链接外部库}
\subsection{库的链接顺序}
\section{使用库的头文件}

\chapter{编译选项}
\section{设置搜索路径}
\subsection{搜索路径的例子}
\subsection{环境变量}
\subsection{扩展搜索路径}
\section{共享库和静态库}
\section{C语言标准}
\subsection{ANSI/ISO}
\subsection{Strict ANSI/ISO}
\subsection{选择指定标准}
\section{警告选项 -Wall}
\section{额外的警告选项}

\chapter{使用预处理器}
\section{宏定义}
\section{带值的宏}
\section{对源文件预处理}

\chapter{为调试而编译}
\section{检查核心文件}
\section{显示反向追踪}

\chapter{编译与优化}
\section{源代码级别优化}
\subsection{普通子表达式的消除}
\subsection{函数内联inling}
\section{时间-空间的权衡}
\subsection{循环展开}
\section{调度规划}
\section{优化级别}
\section{例子}
\section{优化与调试}
\section{优化与编译器警告}

\chapter{编译 C++ 程序}
\section{编译一个简单的 C++ 程序}
\section{使用 C++ 标准库}
\section{模版}
\subsection{使用 C++ 标准库模版}
\subsection{提供自己的模版}
\subsection{明确的模板实例化}
\subsection{export 关键字}

\chapter{特定平台的选项}
\section{Inter 和 AMD x86选项}
\section{DEC Alpha 选项}
\section{SPARC 选项}
\section{POWER/PowerPC 选项}
\section{多架构支持}

\chapter{故障排除}
\section{命令行选项的帮助}
\section{版本号}
\section{Verbose吵闹的编译}

\chapter{编译器相关工具}
\section{使用GNU archiver打包工具创建一个库}
\section{使用 gprof profiler剖析器}
\section{使用 gcov 进行覆盖测试}

\chapter{编译器是如何工作的}
\section{编译过程概述}
\section{预处理器}
\section{编译器}
\section{汇编器}
\section{链接器}

\chapter{检查已编译的文件}
\section{识别文件}
\section{检查符号表}
\section{查找动态链接库}

\chapter{帮助}

\chapter*{后续阅读}
\addcontentsline{toc}{chapter}{后续阅读}
\chapter*{鸣谢}
\addcontentsline{toc}{chapter}{鸣谢}
\chapter*{自由软件组织}
\addcontentsline{toc}{chapter}{自由软件组织}
\chapter*{GNU Free Documentation License}
\addcontentsline{toc}{chapter}{GNU Free Documentation License}
\chapter*{索引表}
\addcontentsline{toc}{chapter}{索引表}


\end{document}

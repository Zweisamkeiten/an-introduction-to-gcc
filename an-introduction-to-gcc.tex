\documentclass[lang=cn,12pt,newtx,scheme=chinese]{elegantbook}

\title{An Introduction to GCC: GCC简明介绍}
\subtitle{for the GNU Compilers \textbf{gcc} and \textbf{g++}}

\author{Brian Gough}
\date{March, 2004}
\bioinfo{Forword}{Richard M. Stallman}

\setcounter{tocdepth}{3}

\cover{cover.jpg}

% 本文档命令
\usepackage{array}
\newcommand{\ccr}[1]{\makecell{{\color{#1}\rule{1cm}{1cm}}}}

% 修改标题页的橙色带
\definecolor{customcolor}{RGB}{32,178,170}
\colorlet{coverlinecolor}{customcolor}
\usepackage{cprotect}

\addbibresource[location=local]{reference.bib} % 参考文献,不要删除

\linespread{1.5} % 行距
\setlength{\parskip}{1em} % 段落间距

%==========================目录层次设置============================%
\setcounter{part}{-1}
\setcounter{chapter}{0}%设置目录计数初值
\setcounter{tocdepth}{3} %设置目录的显示级别。book类从-1开始,为第一级。
\setcounter{secnumdepth}{3}
%(1)设置自动编号的深度(即编号到哪一级别)。{num}在article中取0到5的整数相应于目录中显示到\chapter,\section,\subsection,\subsubsection,\paragraph 或\subparagraph 层次。在book和report中,num取-1到5之间的整数相应于在目录中显示到\part,\chapter,\section,\subseciton,\subsubsection, \paragraph 或\subparagraph 层次。          
%(2)对长标题用\section[abc]{abcdefg}形式的命令。                                            
%(3)另外,还可以利用\addtocounter{secnumdepth}{num}来使得当前章节编号深度增加或减小,num可取正值或负值。
%(4)对于高级内容要求的章节以星号标识,然后在正文中用\begin{advanced}\section{...}\end{advanced}即可
%------------------------------------------------------------------------------------------------------------------------%

\begin{document}

\maketitle
\frontmatter

\tableofcontents
\mainmatter

\chapter*{前言}
\addcontentsline{toc}{chapter}{前言}
\textit{亲爱的理查德.斯托曼, GCC的主要开发者和GNU项目的创建者, 编写了这篇前言.}

\indent 这本书将作为开始使用GCC(the GNU Compiler Collection)的指南. 它将教会你如何使用GCC作为编程工具. 当然, GCC是一个编程工具, 但是它还带有其他更多的. 它是这个二十年来为计算机用户谋求自由的一部分.

\indent 我们都希望有好的软件, 但软件的"好"意味着什么?方便的功能和可靠性是它\textit{技术}意义上的好, 但是这是远远不够的. 好的软件必须要\textit{伦理上}也是好的: 它必须尊重用户的自由.

\indent 作为软件的一个用户, 你应该有以你认为合适的方式运行它的权利, 学习它的源代码然后以你的想法修改它的权利, 重新分发其副本给他人的权利, 发布一个修改的版本以便于你可以帮助社区建设的权利. 当一个程序以这种方式尊重你的自由, 我们可以称其为自由软件. 在 GCC 之前, 有许多其他C编译器, Fortan, Ada, 等等. 但是他们不是自由软件; 你不能自由地使用它们. 我编写了GCC, 因此我们可以使用一个编译器而不用放弃我们的自由.

\indent 为了使用一个计算机系统, 单独一个编译器是不够的, 你需要一整个操作系统. 在1983年, 所有为现代计算机设计的操作系统都是不自由的, 为了弥补这个问题, 在1984年, 我开始开发GNU操作系统, 这是一个 Unix-like 系统, 它将完全包含自由软件. 开发 GCC 就是开发 GNU 的一部分.

\indent 在90年代早期, 接近完成的 GNU 操作系统终于完整了, 通过一个额外的内核, Linux(它在1992年成为了自由软件). 结合体 GNU/Linux 操作系统终于实现了以自由的方式使用计算机的目标. 但是自由从不被自动保障, 因此我们需要为保卫它而工作. 自由软件运动需要你的支持.

\rightline{Richard M. Stallman}
\rightline{February 2004}

\chapter{介绍}
本书的目的是为了介绍\textbf{GNU C}和\textbf{C++}编译器 gcc 和 g++ 的使用. 在阅读完本书后, 你将理解如何编译一个程序, 和如何使用基本的编译器选项来优化和调试. 这本书不会尝试教授C和C++语言本身, 因为这些材料可以在很多其他地方找到 (查看[Further reading], page 91).

那些熟悉其他系统但是第一次接触GNU编译器的有经验的程序员可以跳过前面的章节, '编译一个C程序', '预处理器的使用' 和 '编译一个C++程序'. 剩下的章节将为那些已经知道如何使用其他编译器的人, 提供一个很棒的关于GCC的功能特性的概览.
\section{GCC的简要历史}
\section{GCC的主要特点}
\section{在C和C++中编程}
\section{本手册使用的惯例}

\chapter{编译一个 C 程序}
\section{编译一个简单的 C 程序}
\section{在一个简单的程序中寻找错误}
\section{编译多个源文件}
\section{独立编译文件}
\subsection{从源文件创建目标文件}
\subsection{从目标文件创建可执行文件}
\subsection{目标文件的链接顺序}
\section{链接外部库}
\subsection{库的链接顺序}
\section{使用库的头文件}

\chapter{编译选项}
\section{设置搜索路径}
\subsection{搜索路径的例子}
\subsection{环境变量}
\subsection{扩展搜索路径}
\section{共享库和静态库}
\section{C语言标准}
\subsection{ANSI/ISO}
\subsection{Strict ANSI/ISO}
\subsection{选择指定标准}
\section{警告选项 -Wall}
\section{额外的警告选项}

\chapter{使用预处理器}
\section{宏定义}
\section{带值的宏}
\section{对源文件预处理}

\chapter{为调试而编译}
\section{检查核心文件}
\section{显示反向追踪}

\chapter{编译与优化}
\section{源代码级别优化}
\subsection{普通子表达式的消除}
\subsection{函数内联inling}
\section{时间-空间的权衡}
\subsection{循环展开}
\section{调度规划}
\section{优化级别}
\section{例子}
\section{优化与调试}
\section{优化与编译器警告}

\chapter{编译 C++ 程序}
\section{编译一个简单的 C++ 程序}
\section{使用 C++ 标准库}
\section{模版}
\subsection{使用 C++ 标准库模版}
\subsection{提供自己的模版}
\subsection{明确的模板实例化}
\subsection{export 关键字}

\chapter{特定平台的选项}
\section{Inter 和 AMD x86选项}
\section{DEC Alpha 选项}
\section{SPARC 选项}
\section{POWER/PowerPC 选项}
\section{多架构支持}

\chapter{故障排除}
\section{命令行选项的帮助}
\section{版本号}
\section{Verbose吵闹的编译}

\chapter{编译器相关工具}
\section{使用GNU archiver打包工具创建一个库}
\section{使用 gprof profiler剖析器}
\section{使用 gcov 进行覆盖测试}

\chapter{编译器是如何工作的}
\section{编译过程概述}
\section{预处理器}
\section{编译器}
\section{汇编器}
\section{链接器}
编译的最后一个阶段是链接所有目标文件为一个可执行文件. 在实际情况下, 一个可执行文件需要许多来自系统和C运行时\textbf{(crt)}库的外部函数. 最后, gcc内部实际使用的链接命令非常复杂. 例如, 链接一个 \textit{Hello World} 程序的整个命令如下:
\begin{lstlisting}
  $ ld -dynamic-linker /lib/ld-linux.so.2 /usr/lib/crt1.o /usr/lib/crti.o /usr/lib/gcc-lib/i686/3.3.1/crtbegin.o -L/usr/lib/gcc-lib/i686/3.3.1 hello.o -lgcc -lgcc_eh -lc -lgcc -lgcc_eh /usr/lib/gcc-lib/i686/3.3.1/crtend.o /usr/lib/crtn.o
\end{lstlisting}

  幸运的是我们从不需要直接键入如上的命令--当调用如下时, 整个链接过程已经被 \textbf{gcc} 透明地处理: 

\begin{lstlisting}
  $ gcc hello.o
\end{lstlisting}

  这将把 '\textbf{hello.o}' 目标文件连接到 C 标准库, 并产生一个可执行文件 '\textbf{a.out}':

\begin{lstlisting}
  $ ./a.out
  Hello, world!
\end{lstlisting}

  一个C++程序的目标文件也可以使用一个 \textbf{g++} 命令以相同的方式被链接到 \textbf{C++} 标准库.

\chapter{检查已编译的文件}
\section{识别文件}
\section{检查符号表}
\section{查找动态链接库}

\chapter{帮助}

\chapter*{后续阅读}
\addcontentsline{toc}{chapter}{后续阅读}
\chapter*{鸣谢}
\addcontentsline{toc}{chapter}{鸣谢}
\chapter*{自由软件组织}
\addcontentsline{toc}{chapter}{自由软件组织}
\chapter*{GNU Free Documentation License}
\addcontentsline{toc}{chapter}{GNU Free Documentation License}
\chapter*{索引表}
\addcontentsline{toc}{chapter}{索引表}


\end{document}
